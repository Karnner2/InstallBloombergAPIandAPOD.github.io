\documentclass[]{book}
\usepackage{lmodern}
\usepackage{amssymb,amsmath}
\usepackage{ifxetex,ifluatex}
\usepackage{fixltx2e} % provides \textsubscript
\ifnum 0\ifxetex 1\fi\ifluatex 1\fi=0 % if pdftex
  \usepackage[T1]{fontenc}
  \usepackage[utf8]{inputenc}
\else % if luatex or xelatex
  \ifxetex
    \usepackage{mathspec}
  \else
    \usepackage{fontspec}
  \fi
  \defaultfontfeatures{Ligatures=TeX,Scale=MatchLowercase}
\fi
% use upquote if available, for straight quotes in verbatim environments
\IfFileExists{upquote.sty}{\usepackage{upquote}}{}
% use microtype if available
\IfFileExists{microtype.sty}{%
\usepackage{microtype}
\UseMicrotypeSet[protrusion]{basicmath} % disable protrusion for tt fonts
}{}
\usepackage[unicode=true]{hyperref}
\hypersetup{
            pdftitle={Installing Bloomberg APOD and API},
            pdfauthor={Dr.~Fluharty-Jaidee},
            pdfborder={0 0 0},
            breaklinks=true}
\urlstyle{same}  % don't use monospace font for urls
\usepackage{natbib}
\bibliographystyle{apalike}
\usepackage{longtable,booktabs}
\usepackage{graphicx,grffile}
\makeatletter
\def\maxwidth{\ifdim\Gin@nat@width>\linewidth\linewidth\else\Gin@nat@width\fi}
\def\maxheight{\ifdim\Gin@nat@height>\textheight\textheight\else\Gin@nat@height\fi}
\makeatother
% Scale images if necessary, so that they will not overflow the page
% margins by default, and it is still possible to overwrite the defaults
% using explicit options in \includegraphics[width, height, ...]{}
\setkeys{Gin}{width=\maxwidth,height=\maxheight,keepaspectratio}
\IfFileExists{parskip.sty}{%
\usepackage{parskip}
}{% else
\setlength{\parindent}{0pt}
\setlength{\parskip}{6pt plus 2pt minus 1pt}
}
\setlength{\emergencystretch}{3em}  % prevent overfull lines
\providecommand{\tightlist}{%
  \setlength{\itemsep}{0pt}\setlength{\parskip}{0pt}}
\setcounter{secnumdepth}{5}
% Redefines (sub)paragraphs to behave more like sections
\ifx\paragraph\undefined\else
\let\oldparagraph\paragraph
\renewcommand{\paragraph}[1]{\oldparagraph{#1}\mbox{}}
\fi
\ifx\subparagraph\undefined\else
\let\oldsubparagraph\subparagraph
\renewcommand{\subparagraph}[1]{\oldsubparagraph{#1}\mbox{}}
\fi
\usepackage{booktabs}
\usepackage{amsthm}
\makeatletter
\def\thm@space@setup{%
  \thm@preskip=8pt plus 2pt minus 4pt
  \thm@postskip=\thm@preskip
}
\makeatother

\title{Installing Bloomberg APOD and API}
\author{Dr.~Fluharty-Jaidee}
\date{2020-05-21}

\begin{document}
\maketitle

{
\setcounter{tocdepth}{1}
\tableofcontents
}
\chapter{Forward}\label{forward}

\section{Purpose}\label{purpose}

This manual is intended for use by West Virginia University active
research faculty and PhD students. You must have an active Bloomberg
account associated with West Virginia University in order to access the
Bloomberg license.

The use of Bloomberg Anywhere is temporarily granted during the COVID-19
closure and is expected to be removed at a future date. The APOD will
become unusable once the access is rescinded. Users will still be able
to install the Python API and access the data by using the Terminals
housed in the Department of Finance at the John Chambers College of
Business and Economics.

\section{Bloomberg User Agreement}\label{bloomberg-user-agreement}

You are responsible for reviewing the Bloomberg User Agreement. Some
important points for you to note:

\begin{itemize}
\tightlist
\item
  There can be only 12 concurrent log-ins because West Virginia
  University has 12 individual Terminal licenses. If more than 12 users
  login simultaneously, the 13th user will be informed that she cannot
  log in until another user disconnects. For this reason, it is good
  etiquette to log out of Bloomberg when you are finished using it.
\item
  Bloomberg limits the download amounts in each request via the API
  (either Python or Excel), you may need to chunk your request into
  components.
\item
  You may not use the Terminal or APOD to obtain information that you
  trade on. The West Virginia University license does not allow for
  trades to be placed or made on the information obtained from the
  terminals.
\item
  The license specifies that data obtained from the service may not be
  removed from your local drive. You may not obtain data for any
  individual outside of West Virginia University.
\item
  You may not send data outside of the country without express approval
  from the Office of Export Control at West Virginia University. This is
  a matter of federal law. The Directorate of Defense Trade Controls and
  the Bureau of Industry and Security are responsible for regulating the
  export of information and research related material. Bloomberg data
  services is considered controlled technology. Violations range from
  \$50,000 to \$1,000,000 in fines and up to 10 years in prison.
  Questions on this policy can be found at
  \href{https://exportcontrol.wvu.edu/}{Export Control}.
\end{itemize}

You can review the entire
\href{http://www.bbhub.io/bnef/sites/4/2012/12/Bloomberg_Finance_L1.P-Standard-S.pdf}{Subscription
Terms and Conditions} on BBhub.io.

\chapter{Installing Bloomberg and API}\label{Install}

\section{Getting Started}\label{getting-started}

First, you will need to have a Bloomberg account. To gain access to
Bloomberg, we will first initial a Bloomberg Anywhere you need a
previously authorized account. If you do not yet have an account, you
will need to create one.

\section{The Bloomberg Terminal}\label{the-bloomberg-terminal}

Next, you need to install the
\href{https://www.bloomberg.com/professional/download/2183/}{Bloomberg
Terminal Client}. Install the Terminal Client; this will restart your
computer and may some time.

After installing the Terminal, you will need to activate Bloomberg. Go
to Start \textgreater{} Programs \textgreater{} Bloomberg. The LaunchPad
will begin, and the Terminal Screens will launch. You will be prompted
with a login screen and a request to submit your Access Key.

At this point, you will need to call
\href{https://www.bloomberg.com/professional/support/support-numbers/}{Bloomberg
Customer Support}, and you will need access to your phone and email to
receive security codes.

Call Bloomberg Customer Support and inform them that you would like to
activate an APOD through Bloomberg Anywhere. Your account needs to be a
West Virginia University associated account in order for this to work.
Additionally, your account needs to be in good standing and not expired.
If your account is expired, the representative can send you a
re-activation prompt to your email.

The support specialist should provide you with an Access Key which you
will type into the appropriate place on the Terminal home screen. This
will link your APOD Terminal with the Bloomberg service.

\section{Log-In And Test}\label{log-in-and-test}

After you have successfully created your APOD Terminal creation on your
local, compute you will need to login to test your connection. Please be
aware that if your computer does not have sufficient hardware
requirements, the Terminal may significantly reduce the speed of your
system. The Terminal windows will automatically be adjusted to meet your
resolution specifications, but I find that minimizing 2 or 3 of the
screens can reduce the demands on the system. Otherwise, you should
expect your computer will freeze.

Once you have logged into the Terminal, you should test it to see if the
functions are working properly. Type ``CVX US Equity'' in the search bar
and then GP for graphics plot. This will produce a graphic image of the
historical prices for Chevron, Inc. The scope of this introduction does
not cover the features of Bloomberg, so I encourage you to explore.

\chapter{Installing the Python API}\label{installing-the-python-api}

\section{Requirements}\label{requirements}

First, you will need at least the latest version of Python, 3.7. The
easiest method of installation is to download Anaconda from the Anaconda
website by selecting the
\href{https://www.anaconda.com/products/individual}{Individual Edition}.

Once you have downloaded Anaconda, you will want to launch Anaconda
Navigator. The Navigator is a GUI (graphic user interface similar to the
Bloomberg LaunchPad) which launches and installs various programs housed
under the Anaconda Library which was installed on your local drive.
Please launch Spyder, a Python IDE. I do not suggest the use of Jupiter
Notebook for programmatic coding.

Once Spyder loads, test the system by importing the pandas library.

\begin{verbatim}
# run in python script.py
import pandas as pd 
\end{verbatim}

If the Python console does not report anything and return \textgreater{}
then it was correctly loaded. Next close the entire Spyder we will come
back to it later.

\section{Downloading and Unzipping the Bloomberg
API}\label{downloading-and-unzipping-the-bloomberg-api}

Download the C/C++ supported or experimental release, whichever is
appropriate for your operating system. Find the blp folder in your drive
directory (i.e. \(C:/ blp\)), place the BloombergWindowsSDK (or other OS
SDK) in this directory by unzipping it into this location (you will need
to have unzip, 7zip, or some other related compressor installed.)

From within the Bloomberg C++ SDK folder you have just placed into the
\(C:/blp\) directory locate the \(bin\) folder inside
\(BloombergWindowsSDK/C++API/.../bin/DAPI\) and find the blpapi3\_32.dll
and blpapi3\_64.dll. Copy these driver files and paste them into the
DAPI folder (\(C:/ blp/DAPI\)), and overwrite the driver files that are
in that directory already.

\section{Setting-Up The PATH Link}\label{setting-up-the-path-link}

Go to Start \textgreater{} Search \textgreater{} ``System Variables''
\textgreater{} Environment Variables find Path under `System Variables'
and edit this value. Add a new PATH listing and hit ``Browse''. Locate
the \(C:/ blp/DAPI\) folder and add this to your root PATH.

\section{Installing the API}\label{installing-the-api}

Next go to Start \textgreater{} Search \textgreater{} ``Anaconda
Prompt''. A terminal should begin which will activate conda.

Run the following commands in the terminal -- you may need to type them
in as copying is sometimes disabled.

\begin{verbatim}
pip install blpapi --index-url=https://bloomberg.bintray.com/pip/simple
pip install xbbg
\end{verbatim}

Allow the installation to complete.

\section{Running the API in Python}\label{running-the-api-in-python}

Launch Spyder again by going to Start \textgreater{} Programs
\textgreater{} Anaconda \textgreater{} Anaconda Navigator then launching
Spyder (alternatively search for Spyder in the search box).

Import the xbbg library for Python.

\begin{verbatim}
from xbbg import blp
\end{verbatim}

If the console returns \textgreater{} it was imported without an error
(cross your fingers). At this point you need to have launched and signed
into the Bloomberg Terminal.

\section{Your First Test: Download Chevron's Daily
Prices}\label{your-first-test-download-chevrons-daily-prices}

\begin{verbatim}
blp.bdh( tickers='CVX US Equity', flds=[ 'Last_Price'], 
start_date='2018-10-10', end_date='2020-05-20',
)
\end{verbatim}

If the console returns a pandas dataframe of prices you have
successfully completed your first pull from Bloomberg.

\chapter{Python API Commands}\label{python-api-commands}

Multiple Python API Clients connect to Bloomberg. There is the direct
and strict BLP. \textbf{BLP} will return responses to you in a regular
JSON style API response. You would need to know how to parse the JSON or
use JSON or JSONLight to extract the information you are looking for.

\textbf{pdblp} and \textbf{xbbg} are wrappers for the BLP. These will
provide pandas data frames ready for you to export to csv or excel or
analyze directly in Python or other statistical analytics software.

I suggest you make use of \textbf{pdblp} or \textbf{xbbg}, reserve the
use of BLP when the pdblp or xbbg do not provide you the functionality
you require. For example, both pdblp and xbbg packages have limited
field search capabilities while //blp/searchFieldRequest allows you to
programmatically search fields similar to how you would with the FLDS
command in the Bloomberg Terminal. For the purpose of this tutorial, I
will show the most common commands in the \textbf{xbbg} as these produce
results similar to those found in excel and it is the most
straightforward. I encourage you to review the \textbf{pdblp} package on
your own as it is very similar but provides some more features.

\section{XBBG}\label{xbbg}

The documentation for the \textbf{XBBG} package can be found in its
python or anaconda library source on the internet. It also has a
\href{https://xbbg.readthedocs.io/en/latest/}{read-the-docs page} which
I find to be the simplest. Much of the code here is taken from that
source.

There are three main reference commands similar to the commands we run
in the Excel API. BDP -- data point request, BDH -- historical request,
and BDS -- block requests which return groups of information about the
field (not all fields have this capability). Lastly, there is the BDIB
request which is for intraday trading information up-to tick level
frequency. Be aware that intraday trading information is available
through the Excel API for the past 140 trading days and through the
Python API for the past 240 trading days (approximately 1 year).

\subsection{A sample BDP Command}\label{a-sample-bdp-command}

The simplest command is the BDP or Bloomberg Data Point; it will return
a single point in time of data or a static data point which never
changes. Let's try it out on Spyder.

\begin{verbatim}
blp.bdp(tickers='CVX US Equity', flds=['Security_Name', 'GICS_Sector_Name'])
\end{verbatim}

\subsection{A sample BDH Command}\label{a-sample-bdh-command}

The most useful command for finance research will likely be the BDH or
Bloomberg Data Historical, which returns you a set of at most daily
values for variables which have historical data. We have already done an
example of these, but there is another:

\begin{verbatim}
blp.bdh(tickers='SPX Index', flds=['High', 'Low', 'Last_Price'],
start_date='2018-10-10', end_date='2018-10-20',
)
\end{verbatim}

Please note that while it may seem that data would not be reported under
the historical feature because we tend to think of prices and returns in
a time series and not much else this is not true. There are some 62,000
firm-level variables that Bloomberg allows access to which have
historical data.

\subsection{A Sample BDS Command}\label{a-sample-bds-command}

The BDS command can be useful in finding information that Bloomberg
reports in a table or on a pre-made screen which you would like
programmatic access to. Unfortunately, they are unreliable, and the
syntax is complicated and not uniform by field. Since there are so many
fields to deal with these are less efficient. The following example
provides the history of all dividends paid from Chevron over January to
June of 2018.

\begin{verbatim}
blp.bds('CVX US Equity', 'DVD_Hist_All', DVD_Start_Dt='20180101', DVD_End_Dt='20180531')
\end{verbatim}

\subsection{A sample BDIB Command}\label{a-sample-bdib-command}

The BDIB command provides intraday prices. You are limited to the
historical value of prices you can extract and the amount of values.
Generally, Bloomberg will not allow you to extract more than three to
four thousand rows of information at a given time so you may need to
break this information up to get it to work correctly.

\begin{verbatim}
blp.bdib(ticker='BHP AU Equity', dt='2018-10-17')
\end{verbatim}

\section{Extracting Data Out of
Python}\label{extracting-data-out-of-python}

I suggest you get familiar with loops, arrays, dataframes (pandas), and
the csv package in Python. You can directly export data by writing it to
a dataframe and them from the pandas package you can write that
dataframe to a csv file in your current working directory. It would be
best if you changed your working directory at the start of your Spyder
session it will default to a place you typically do not want to place
files.

\bibliography{book.bib}

\end{document}
